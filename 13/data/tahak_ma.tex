\documentclass[8pt, a4paper]{extarticle}
 
\usepackage[utf8]{inputenc}
\usepackage[hmargin=1cm,vmargin=2cm]{geometry}
\usepackage[fleqn]{amsmath}
\usepackage{amsfonts}
\usepackage{amssymb}
\usepackage{multicol}
 
\newcommand{\arctg}{\textup{arctg}}
\newcommand{\arccotg}{\textup{arccotg}}
\newcommand{\arcsinh}{\textup{arcsinh}}
\newcommand{\diff}{\textup{d}}
\newcommand{\reals}{\mathbb{R}}
 
 
\begin{document}
\begin{multicols}{3}
\textbf{Goniometrické identity}
\begin{align*}
\sin x = -\sin(-x) \\
\cos x = \cos(-x) \\
\sin(x+\pi)=-\sin(x) \\
\cos(x+\pi)=-\cos(x) \\
\sin\left(x+\frac{\pi}{2}\right)=\cos x \\
\cos\left(x+\frac{\pi}{2}\right)=-\sin x \\
\sin^2x+\cos^2x=1 \\
\sin(x+y)=\sin x\cos y + \cos x\sin y \\
\cos(x+y)=\cos x\cos y - \sin x\sin y \\
\sin x + \sin y = 2\sin\left(\frac{x+y}{2}\right)\cos\left(\frac{x-y}{2}\right) \\
\sin x - \sin y = 2\cos\left(\frac{x+y}{2}\right)\sin\left(\frac{x-y}{2}\right) \\
\cos x + \cos y = 2\cos\left(\frac{x+y}{2}\right)\cos\left(\frac{x-y}{2}\right) \\
\cos x - \cos y = -2\sin\left(\frac{x+y}{2}\right)\sin\left(\frac{x-y}{2}\right) \\
\sin^2x=\frac{1-\cos(2x)}{2} \\
\cos^2x=\frac{1+\cos(2x)}{2} \\
\frac{1}{\cos^2x} = 1 + \tan^2x\\
\arccotg x = \arctg\left(\frac{1}{x}\right)\\
\cosh^2(x) - \sinh^2(x) = 1\\
\cosh(x) = \frac{e^x + e^{-x}}{2}\\
\sinh(x) = \frac{e^x - e^{-x}}{2}\\
\arcsinh(x) = \log(x+\sqrt{x^2+1})
\end{align*}
 
\textbf{Hodnoty gonio. funkcí}
\begin{center}
\begin{tabular}{|c|c|c|c|c|c|c|}
\hline
$\alpha (deg)$ & $0$ & $30$ & $45$ & $60$ & $90$ & $180$ \\ \hline
$\alpha (rad)$ & $0$ & $\frac{\pi}{6}$ & $\frac{\pi}{4}$ & $\frac{\pi}{3}$ & $\frac{\pi}{2}$ & $\pi$ \\ \hline
$\sin\alpha $ & $0$ & $\frac{1}{2}$ & $\frac{\sqrt{2}}{2}$ & $\frac{\sqrt{3}}{2}$ & $1$ & $0$ \\ \hline
$\cos$ & $1$ & $\frac{\sqrt{3}}{2}$ & $\frac{\sqrt{2}}{2}$ & $\frac{1}{2}$ & $0$ & $-1$ \\ \hline
$\tan$ & $0$ & $\frac{\sqrt{3}}{3}$ & $1$ & $\sqrt{3}$ & $N/A$ & $0$ \\ \hline
$\cot$ & $N/A$ & $\sqrt{3}$ & $1$ & $\frac{\sqrt{3}}{3}$ & $0$ & $N/A$ \\ \hline
\end{tabular}
\end{center}
 
\textbf{Exponenciela a logaritmus}
\begin{align*}
x^y=e^{y\log x} \\
\log_yx=\frac{\log x}{\log y} \\
\end{align*}
 
\textbf{Taylorova řada}
Pro $f(x)$ v $a$:\\ $\sum\limits_{n=0}^\infty \frac{f^{(n)}(a)}{n!}(x-a)^n$\\
 
\textbf{Limity}
\begin{align*}
\lim\limits_{x\rightarrow\infty} \sqrt[x]{x} = 1 \\
\lim\limits_{x\rightarrow 0} \frac{\sin x}{x} = 1 \\
\lim\limits_{x\rightarrow 0} \frac{1 - \cos x}{x^2} = \frac{1}{2} \\
\lim\limits_{x\rightarrow 0} \frac{e^x -1}{x} = 1 \\
\lim\limits_{x\rightarrow 0} \frac{\log(x+1)}{x} = 1 \\
\lim\limits_{x\rightarrow 1} \frac{\log(x)}{x-1} = 1 \\
e^x=\lim_{x\to\infty}(1+\frac{x}{n})^n \\
a>0, b\in \mathbb{R}: \lim\limits_{x\rightarrow \infty} x^a\log^bx = \infty \\
a<0, b\in \mathbb{R}: \lim\limits_{x\rightarrow \infty} x^a\log^bx = 0 \\
a>0, b\in \mathbb{R}: \lim\limits_{x\rightarrow 0} x^a\log^bx = 0 \\
a<0, b\in \mathbb{R}: \lim\limits_{x\rightarrow 0} x^a\log^bx = \infty \\
a>0, b\in \mathbb{R}: \lim\limits_{x\rightarrow \infty} \frac{e^{ax}}{x^b} = \infty \\
a<0, b\in \mathbb{R}: \lim\limits_{x\rightarrow \infty} \frac{e^{ax}}{x^b} =  \\
\end{align*}
 
\textbf{Derivace}
\begin{align*}
(f+g)'(a)=f'(a)+g'(a)\\
(f\cdot g)'(a)=f'(a)g(a)+f(a)g'(a) \\
\left(\frac{f}{g}\right)'(a)=\frac{f'(a)g(a)-f(a)g'(a)}{g^2(a)} \\
(f^{-1})'(f(a))=\frac{1}{f'(a)} \\
(f\cdot g)^{(n)}(a)=\sum\limits_{k=0}^n \binom{n}{k}f^{(k)}(x)g^{(n-k)}(a) \\
f'(a):=\lim\limits_{h\to 0}\frac{f(a+h)-f(a)}{h} \\
\end{align*}
 
\textbf{Tabulkové derivace}
\begin{align*}
(x^k)'=kx^{k-1}\\
(e^x)'=e^x\\
(\log|x|)'=\frac{1}{x}\\
(a^x)'=a^x \log a\\
(\log_a x)'=\frac{1}{x\log a}\\
(\sin x)'= \cos x\\
(\cos x)'= -\sin x\\
(\text{tg } x)'= \frac{1}{\cos^2x}\\
(\text{cotg }x)'=\frac{-1}{\sin^2x}\\
(\arcsin x)'=\frac{1}{\sqrt{1-x^2}}\\
(\arccos x)'=\frac{-1}{\sqrt{1-x^2}}\\
(\text{arctg } x)'=\frac{1}{1+x^2}\\
(\text{arccotg }x)'=\frac{-1}{1+x^2} 
\end{align*}
 
\textbf{Limity posloupnosti}
\begin{align*}
\frac{a_{n+1}}{a_n}<1 \Rightarrow K
\end{align*}
 
\textbf{Průběh funkce}
\begin{enumerate}
\item $D(f), H(f)$
\item Limity v krajních bodech
\item Derivace
\item Monotonie
\item Extrémy: $f'(x)=0$
\item Druhá derivace
\item Konvexita $f''(x)\ge0$, konkávnost $f''(x)\le0$
\item Inflexní body
\item Asymptoty $y=kx+q \Leftrightarrow 
\lim\limits_{x \rightarrow\infty}\frac{f(x)}{x}=k,
\lim\limits_{x \rightarrow\infty}(f(x)-kx)=q$
\item Graf
\end{enumerate}
 
 
\textbf{Řady}
 
Nutná podmínka: $\sum a_n < \infty \Rightarrow a_n\rightarrow0$ \\
D'Alambert: $\forall a_n>0: \frac{a_{n+1}}{a_n}<1 \Rightarrow \\ \sum a_n < \infty$ \\
Limitně: $\forall a_n \ge 0, b_n >0: c:=\lim\limits_{x \rightarrow\infty}\frac{a_n}{b_n}$
\begin{enumerate}
\item $0<c<\infty: \sum a_n < \infty \Leftrightarrow \sum b_n < \infty$
\item $c=0: \sum b_n < \infty \Rightarrow \sum a_n < \infty$
\item $c=\infty: \sum b_n = \infty \Rightarrow \sum a_n = \infty$
Cauchy: $\lim\limits_{n \rightarrow\infty}\sqrt[n]{a_n}\le 1 \Rightarrow \sum a_n < \infty$
$\lim\limits_{n \rightarrow\infty}\sqrt[n]{a_n}\ge 1 \Rightarrow \sum a_n = \infty$\\
Leibnitz (neabsolutní): $a_n\rightarrow 0 \Rightarrow \sum (-1)^{n+1}a_n\le \infty$
\end{enumerate}
 
 
\textbf{Ultimátní goniometrická substituce}
\begin{itemize}
\item $ y=\tan\left(x/2\right) $
\item $ dx=\frac{2dy}{1+y^2} $
\item $ \cos x = \frac{1-y^2}{1+y^2} $
\item $ \sin x = \frac{2y}{1+y^2} $
\end{itemize}
\textbf{Známe integrály}
\begin{align*}
\int\limits_{0}^{\pi}\sin^2x=\frac{\pi}{2}\\
\int u\diff v = uv-\int v \diff u\\
\int \frac{x}{1+x} = x - \log{(1+x)}\\
\int \frac{x}{1-x} = -x - \log{(1-x)}
\end{align*}
\textbf{Tipy na substituci}
\begin{itemize}
\item $ \sqrt{1-x^2} \ldots x=\sin(t)$
\item $ \sqrt{x^2-1} \ldots x=\cosh(t)$
\item $ \sqrt{x^2+1} \ldots x=\sinh(t)$
\end{itemize}
\textbf{Aplikace}
\begin{itemize}
\item délka: $l(\varphi)=\int\limits_{a}^{b}\sqrt{x'(t)^2+y'(t)^2+\ldots}\diff t$
\item plocha: $S=\int\limits_{a}^{b}(g-f)\diff x$
\item objem: $V= \pi \int\limits_{a}^{b}f^2(x)\diff x$
\item povrch: $S= 2\pi \int\limits_{a}^{b}f(x)\sqrt{1+f'(x)^2}\diff x$
\end{itemize}
\textbf{Konvergence}
\begin{align*}
\int\limits_{a}^{b}\frac{1}{(b-x)^p} K \Leftrightarrow p < 1 \\
\int\limits_{a}^{b}\frac{1}{(x-a)^p} K \Leftrightarrow p < 1
\end{align*}
 
 
\textbf{Více proměnných}\\
Parciální derivace jsou zaměnitelné pouze na spojitém okolí (většinou polynomy)\\
Diferenciál v $a$: $L(h_1,...,h_n)=\frac{\partial f}{\partial x_1}(a)h_1+...+\frac{\partial f}{\partial x_n}(a)h_n$\\
$\lim\limits_{h\to 0}\frac{f(a+h)-f(a)-L(h)}{||h||}=0$\\
Tečná nadrovina v $a$: $x_{n+1}=f(a)+\frac{\partial f}{\partial x_1}(a)(x_1-a_1)+...+\frac{\partial f}{\partial x_n}(a)(x_n-a_n)$\\
Směrová derivace: $D_vf(a)=\lim\limits_{t\to 0}\frac{f(a+tv)-f(a)}{t}$\\
$\nabla f(a)=\left(\frac{\partial f}{\partial x_1}(a), ..., \frac{\partial f}{\partial x_n}(a)\right)$ \\
$f: \reals^n\to\reals^m, g: \reals^m\to\reals^s: (g\circ f)'(a)=g'(f(a))\circ f'(a)$\\
\textbf{Tipy na limity}
\begin{itemize}
\item $y = 0$
\item $y = kx \ldots$(často $k=1$)
\item $y = x^2$
\item $y = \frac{1}{x}$
\item převod na pol. souřadnice
\end{itemize}
 
\textbf{Převod na polární souřadnice}
\begin{align*}
\lim\limits_{(x,y) \rightarrow (a_x, a_y)} f(x,y)\\
\Downarrow \\
x = a_x + r \cdot \cos{\varphi}\\
y = a_y + r \cdot \sin{\varphi}\\
\Downarrow \\
\lim\limits_{r \rightarrow 0} f(a_x + r \cdot \cos{\varphi}, a_y + r \cdot \sin{\varphi})
\end{align*}
 
\textbf{Per partes}
\begin{align*}
\int f(x)g'(x) = f(x)g(x) - \int f'(x)g(x)\\
\int\limits_{a}^{b} f(x)g'(x) = [f(x)g(x)]_{a}^b- \int\limits_{a}^{b} f'(x)g(x)
\end{align*}
 
\textbf{Substituce}
\begin{align*}
\int f(\varphi(x))\varphi'(x) = \int f(x)\\
\int\limits_{a}^{b} f(\varphi(x))\varphi'(x) = \int\limits_{\varphi(a+)}^{\varphi(b-)} f(x)\\
\end{align*}
 
\textbf{Eul. subst. 1. druhu}
\begin{align*}
\int R(x, \sqrt{\frac{ax+b}{cx+d}}) \rightarrow y=\sqrt{\frac{ax+b}{cx+d}}\\
\end{align*}
 
\textbf{Eul. subst. 2. druhu}
\begin{align*}
\int \frac{1}{x\sqrt{ax^2+bx+c}} \rightarrow t+x=\sqrt{ax^2+bx+c}
\end{align*}
 
%TODO: vyřešit formátování, tohle je hnus
$\\\\\\\\$
 
\textbf{Derivace implic. funkcí}
\begin{itemize}
\item vyjádřit vhodnou funkci pomocí zbylých proměnných
\item zderivovat každou stranu zvlášť (a to podle všech zbylých proměnných)
\item postupně vyjádřit derivace
\item popř. zderivovat znovu (z rovnice, ne z vyjádření)
\end{itemize}
 
\textbf{Jednostranná derivace (snad)}
\begin{align*}
\frac{\partial f}{\partial x_i}(a) = \lim\limits_{x_i \rightarrow a_i} f(a_1, \ldots x_i, \ldots a_n)
\end{align*}
 
\textbf{Vázané extrémy}
\begin{align*}
Funkce f(x,y,z); mnozina M\\
M=\{g_1(x,y,z)<1 \ldots g_n(x,y,z)<0\}\\
\nabla f = \sum\limits_{i=1}^n \lambda_i \nabla g_i
\end{align*}
 
\textbf{Extrémy}
\begin{enumerate}
\item Všechny parciální derivace 1. stupně
\item Stacionnární body (vyřešit soustavu rovnic parciálních derivacích položených nule)
\item Všechny parciální derivace 2. stupně
\item Pro každý stac. bod sestavit Jacobiho matici
\item Otestovat pos. (neg.) definitnost $\rightarrow$ minimum (maximum)
\end{enumerate}
 
\end{multicols}
 
\begin{flushbottom}
 
\begin{center}
Hodně štěstí!\\
© 2013 - Pavel Kalvoda, Petr Bělohlávek\\
MIT license\\
\vspace{2cm}
{\LARGE http://prints.os1.cz/13/}
\end{center}
\end{flushbottom}
\end{document}
